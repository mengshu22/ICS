\documentclass[a4paper,12pt]{article}     %页面大小和字体大小
\usepackage{ctex}
\usepackage{geometry}
\usepackage{fancyvrb}
\usepackage{amsmath}
\usepackage{graphicx}
\usepackage[T1]{fontenc}
\geometry{left=2.0cm, right=2.0cm, top=3.0cm, bottom=3.0cm}   %页边距
\linespread{1.5}      %行距

\begin{document}

\begin{center}   %居中设置
孟澍 \ 3210101819
\end{center}


\noindent %顶格(不缩进)
\textbf{5.15}\\
R1 = x3121.\\
R2 = x4566.\\
R3 = xABCD.\\
R4 = xABCD.\\

% ~\\
% \textbf{5.16}\\
% a. PC-relative mode. The reason is that 
% b. \ 7. Because $2^7 = 128 > 120.$\\
% c. \ 3. Because $32 - 8 - 3 \times 7 = 3.$\\

~\\
\textbf{5.37}\\
PC -> MAR -> MEMORY -> MDR -> IR -> SEXT[8:0] -> ADDR1MUX -> ADDR2MUX -> ADDER -> MARMUX -> MAR -> MEMORY -> MDR -> MAR -> MEMORY -> MDR -> LOGIC \& REGFILE\\

~\\
\textbf{5.39}\\
PC -> MAR -> MEMORY -> MDR -> IR -> SEXT[8:0] -> ADDR1MUX -> ADDR2MUX -> ADDER -> MARMUX -> REGFILE

~\\
\textbf{5.50}\\
BR: PC.\\
LEA: R7.\\
LD: MAR.\\

\newpage
\textbf{6.9}\\
\begin{Verbatim}[frame = single, numbers = left]
0011 0000 0000 0000    ; x3000
0010 000 000000101     ; R0 <- m[x3007]
0010 001 000000101     ; R1 <- m[x3008]
1111 0000 0010 0001    ; TRAP x21
0001 001 001 1 11111   ; R1 <- R1 - 1
0000 001 111111101     ; BRp #-3
1111 0000 0010 0101    ; HALT
0000 0000 0101 1010    ; ASCII code of 'Z'
0000 0000 0110 0100    ; #100
\end{Verbatim}

~\\
\textbf{6.10}\\
\begin{Verbatim}[frame = single, numbers = left]
0011 0000 0000 0000    ; x3000
0101 001 010 1 00001   ; R1 <- R2 AND #1
0000 001 000000011     ; BRz #3
0010 000 000000101     ; R0 <- m[x3006]
1111 0000 0010 0001    ; TRAP x21
1111 0000 0010 0101    ; HALT
0010 000 000000011     ; R0 <- m[x3009]
1111 0000 0010 0001    ; TRAP x21
1111 0000 0010 0101    ; HALT
0000 0000 0101 1001    ; ASCII code of 'Y'
0000 0000 0100 1110    ; ASCII code of 'N'
\end{Verbatim}

\end{document}
