\documentclass[a4paper,12pt]{article}     %页面大小和字体大小
\usepackage{ctex}
\usepackage{geometry}
\usepackage{mathptmx}
\usepackage{amsmath}
\usepackage{graphicx}
\usepackage[T1]{fontenc}
\geometry{left=2.0cm, right=2.0cm, top=3.0cm, bottom=3.0cm}   %页边距
\linespread{1.5}      %行距

\begin{document}

\begin{center}   %居中设置
孟澍 \ 3210101819
\end{center}

\noindent %顶格(不缩进)
\textbf{4.3}\\
The program counter is actually not a counter. Instead of counting something, it just stores the address of the instruction to be processed next. Meanwhile the name "instruction pointer" is better because it feels like that this register is just "pointing" to the next instruction to be processed.\\

~\\
\textbf{4.8}\\
a. \ 8. Because $2^8 = 256 > 225.$\\
b. \ 7. Because $2^7 = 128 > 120.$\\
c. \ 3. Because $32 - 8 - 3 \times 7 = 3.$\\

~\\
\textbf{4.9}\\
PC $\leftarrow$ PC + 1.\\

~\\
\textbf{4.19}\\
a. \ MAR: 010 \quad MDR: 01010000\\
b. \ MDR: 00111001 \\

~\\
\textbf{5.1}\\
ADD: operate instruction, can use immediate and register addressing mode.\\
JMP: control instruction, can only use register addressing mode.\\
LEA: operate instruction, can only use PC-relative mode.\\
NOT: operate instruction, can only use register addressing mode.\\

\newpage
\noindent
\textbf{5.4}\\
a. \ 8. Because $2^8 = 256.$\\
b. \ 6. Because a 6-bit 2's complement integer can represent the values ranged from $-2^{5} = -32$ to $2^{5}-1 = 31.$\\
c. \ 6. Because $10 - (3+1) = 6.$\\

~\\
\textbf{5.9}\\
Only c could be used for NOP. In the excution phase of a, the ALU is used to add 0 to R1, and the CC is setted, so it does something. B will skip the instruction behind it in the memory so the program cannot work correctly. C just do nothing. Therefore, only c could be used for NOP.\\
Only the ADD instruction uses the ALU and set CC.\\
\end{document}
