\documentclass[a4paper,12pt]{article}     %页面大小和字体大小
\usepackage{ctex}
\usepackage{ulem}
\usepackage{geometry}
\usepackage{fancyvrb}
\usepackage{amsmath}
\usepackage{graphicx}
\usepackage[T1]{fontenc}
\geometry{left=2.0cm, right=2.0cm, top=3.0cm, bottom=3.0cm}   %页边距

\begin{document}

\begin{center}   %居中设置
孟澍 \ 3210101819
\end{center}


\noindent %顶格(不缩进)
\textbf{8.1}\\
Last in, First out. That is, the last thing you stored in the stack is the first thing you remove from it.

~\\
\textbf{8.7}\\
\begin{Verbatim}[numbers = left]
;
; Subroutines for carrying out the PUSH and POP functions to handle
; elements of arbitrary sizes. This program works with a stack consisting
; of memory locations x3FFF through x3FFB. R6 is the stack pointer. R3 is
; the size of the element. R4 is the address of the start of the element.
;
POP           ST R0,Save0
              ST R1,Save1       ; Save registers that
              ST R2,Save2       ; are needed by POP
              LD R1,EMPTY       ; EMPTY contains -x4000
              ADD R5,R3,R6      ; R5 is the address of SP after POP
              ADD R2,R5,R1      ; Compare stack pointer to x4000
              BRp fail_exit     ; Branch if stack is underflow
              ADD R4,R4,R3      ; Move the R4 to the end
              ADD R4,R4,#-1     ; of the element
;
              LDR R0,R6,#0      ; The actual "pop"
              STR R0,R4,#0      ; Store the element
              ADD R4,R4,#-1     ; The next address
              ADD R6,R6,#1      ; Adjust stack pointer
              ADD R3,R3,#-1     ; Count the size
              BRp #-6           ; Continue loop
              BRnzp success_exit
;
PUSH          AND R5,R5,#0
              ST R0,Save0
              ST R1,Save1       ; Save registers that
              ST R2,Save2       ; are needed by PUSH
              LD R1,FULL        ; FULL contains -x3FFB
              NOT R5,R3
              ADD R5,R5,1
              ADD R5,R5,R6      ; R5 is the address of SP after PUSH
              ADD R2,R5,R1      ; Compare stack pointer to x3FFB
              BRn fail_exit     ; Branch if stack is overflow
;
              ADD R6,R6,#-1     ; Adjust stack pointer
              LDR R0,R4,#0
              STR R0,R6,#0      ; The actual "push"
              ADD R4,R4,#1      ; The next address
              ADD R3,R3,#-1     ; Count the size
              BRp #-6           ; Continue loop
;
success_exit  LD R2,Save2       ; Restore original
              LD R1,Save1       ; register values
              LD R0,Save0
              AND R5,R5,#0      ; R5 <-- success
              RET
;
fail_exit     LD R2,Save2 ; Restore original
              LD R1,Save1 ; register values
              LD R0,Save0
              AND R5,R5,#0
              ADD R5,R5,#1 ; R5 <-- failure
              RET
;
EMPTY         .FILL xC000 ; EMPTY contains -x4000
FULL          .FILL xC005 ; FULL contains -x3FFB
Save0         .FILL x0000
Save1         .FILL x0000
Save2         .FILL x0000
\end{Verbatim}


\newpage
\noindent
\textbf{8.8}\\
~\\
a:
\begin{tabular}{|c|}
  F\\
  \hline
  A\\
  \hline
\end{tabular}\\
~\\
b: \ After PUSH D or after PUSH E.\\
~\\
c:
\begin{tabular}{|c|}
  M\\
  \hline
  F\\
  \hline
  A\\
  \hline
\end{tabular}\\

~\\
~\\
\textbf{8.12}\\
\begin{minipage}{\textwidth}
  \begin{minipage}[t]{0.45\textwidth}
    \centering
    \begin{tabular}{c|c|}
      \cline{2-2}
      x4000 & x0041\\
      \cline{2-2}
      x4001 & xA243\\
      \cline{2-2}
      x4002 & xA243\\
      \cline{2-2}
      x4003 & xBBBB\\
      \cline{2-2}
      x4004 & x3100\\
      \cline{2-2}
      x4005 & x3100\\
      \cline{2-2}
    \end{tabular}\\
    ~\\
    ~\\
    \begin{tabular}{c|c|}
      \cline{2-2}
      x4100 & x0043\\
      \cline{2-2}
      x4101 & xBBBB\\
      \cline{2-2}
      x4102 & xA243\\
      \cline{2-2}
      x4103 & xA243\\
      \cline{2-2}
      x4104 & xBBBB\\
      \cline{2-2}
    \end{tabular}\\
    ~\\
    ~\\
    \begin{tabular}{c|c|}
      \cline{2-2}
      x3100 & x0045\\
      \cline{2-2}
      x3101 & x0000\\
      \cline{2-2}
      x3102 & x4000\\
      \cline{2-2}
      x3103 & x4000\\
      \cline{2-2}
      x3104 & xBBBB\\
      \cline{2-2}
    \end{tabular}\\
  \end{minipage}
  \begin{minipage}[t]{0.45\textwidth}
    \vspace{-5em}
    \centering
    \begin{tabular}{c|c|}
      \cline{2-2}
      x3050 & x4000\\
      \cline{2-2}
    \end{tabular}\\
    ~\\
    ~\\
    \begin{tabular}{c|c|}
      \cline{2-2}
      xA243 & x0042\\
      \cline{2-2}
      xA244 & x4100\\
      \cline{2-2}
      xA245 & x4000\\
      \cline{2-2}
      xA246 & xBBBB\\
      \cline{2-2}
      xA247 & x4100\\
      \cline{2-2}
      xA248 & x4100\\
      \cline{2-2}
    \end{tabular}\\
    ~\\
    ~\\
    \begin{tabular}{c|c|}
      \cline{2-2}
      xBBBB & x0044\\
      \cline{2-2}
      xBBBC & x3100\\
      \cline{2-2}
      xBBBD & x3100\\
      \cline{2-2}
      xBBBE & x4100\\
      \cline{2-2}
      xBBBF & xA243\\
      \cline{2-2}
      xBBC0 & x4000\\
      \cline{2-2}
    \end{tabular}\\
  \end{minipage}
\end{minipage}


\newpage
\noindent
\textbf{8.14}\\
\linespread{1.2}
\begin{Verbatim}[commandchars=\\\{\}]
.ORIG   x3000
        LEA R3, NUM1
        LEA R4, NUM2
        LEA R5, RESULT
        LDR R1, R3, #0
        LDR R2, R4, #0
        ADD R0, R1, R2
        STR R0, R5, #0
        \uline{JSR X}          (a)
        LDR \uline{R1, R3, #1} (b)
        LDR \uline{R2, R4, #1} (c)
        ADD R1, \uline{R1, R2} (d)
        \uline{ADD R0, R0, R1} (e)
        STR R0, \uline{R5, #1} (f)
        TRAP x25

X       ST R4, SAVER4
        AND R0, R0, #0
        AND R4, R1, R2
        BRn \uline{LABEL}      (g)
        ADD R1, R1, #0
        BRn \uline{ADDING}     (h)
        ADD \uline{R2, R2, #0} (i)
        BRn ADDING
        BRnzp EXIT
ADDING  ADD R4, R1, R2
        BRn EXIT
LABEL   ADD R0, R0, #1
EXIT    LD R4, SAVER4
        RET

NUM1    .BLKW 2
NUM2    .BLKW 2
RESULT  .BLKW 2
SAVER4  .BLKW 1
\end{Verbatim}
\end{document}